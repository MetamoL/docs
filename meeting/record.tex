\documentclass[12pt]{jsarticle}
\usepackage[dvipdfmx]{graphicx}
\textheight = 25truecm
\textwidth = 18truecm
\topmargin = -1.5truecm
\oddsidemargin = -1truecm
\evensidemargin = -1truecm
\marginparwidth = -1truecm

\def\theenumii{\Alph{enumii}}
\def\theenumiii{\alph{enumiii}}
\def\labelenumi{(\theenumi)}
\def\labelenumiii{(\theenumiii)}
\def\theenumiv{\roman{enumiv}}
\def\labelenumiv{(\theenumiv)}
%\usepackage{comment}

%%%%%%%%%%%%%%%%%%%%%%%%%%%%%%%%%%%%%%%%%%%%%%%%%%%%%%%%%%%%%%%%
%% sty/ にある研究室独自のスタイルファイル
\usepackage{jtygm}  % フォントに関する余計な警告を消す
\usepackage{nutils} % insertfigure, figref, tabref マクロ

\def\figdir{./figs} % 図のディレクトリ
\def\figext{pdf}    % 図のファイルの拡張子

\begin{document}
%%%%%%%%%%%%%%%%%%%%%%%%%%%%
%% 表題
%%%%%%%%%%%%%%%%%%%%%%%%%%%%
\begin{center}
  {\LARGE 記録書 No.3}\\
  {\verb|(2018/04/24 ~ 2018/05/14|)}
\end{center}

\begin{flushright}
  2018/5/14\\
  吉田 修太郎 
\end{flushright}
%%%%%%%%%%%%%%%%%%%%%%%%%%%%
%% 概要
%%%%%%%%%%%%%%%%%%%%%%%%%%%%
\section{前回ミーティングからの指導・指摘事項}
\label{sec:introduction}
\begin{enumerate}
\item .(4/4, 104号室,乃村先生)
\end{enumerate}

\section{実績}\label{jisseki}
\subsection{研究関連}\label{kenkyuu}
\begin{enumerate}
\item 2018年度B4新人研修課題に関する項目
  \begin{enumerate}
  \item Debianのインストール(100\% +100\%)
  \item Linuxカーネルの再構築(100\% +100\%)
  \item システムコールの実装(100\% +100\%)
  \item システムコール実装の手順書作成(100\% +20\%)
  \item RubyによるSlackBotプログラムの作成(100\% +85\%)
  \item RubyによるSlackBotプログラムの報告書作成(100\% +100\%)
  \end{enumerate}
  
\end{enumerate}
\subsection{研究室関連}\label{kenkyuushitu}
\begin{enumerate}
\item (4/2) 平成30年度SWLAB新B4ガイダンス
\item (4/3) 第349回New打ち合わせ
\item (4/3) Web勉強会
\item (4/3) rbenv勉強会
\item (4/4) Git勉強会
\item (4/4) 乃村研お花見
\item (4/18) 第350回New打ち合わせ
\item (4/19) 乃村研ミーティング
\end{enumerate}
\subsection{大学関連}\label{daigaku}
\begin{enumerate}
\item (4/23,24,5/1,7) 非手続き型言語
\end{enumerate}

\section{詳細および反省・感想}
\subsection{研究関連}
\begin{description}
\item[(1.E)] 4/20現在,RubyによるSlackBotプログラムの作成に取り組んでいる.今回が初めてのRubyを用いてのコーディングであるので,よく言語仕様を調べながらコーディングを進める.
\end{description}
\subsection{研究室関連}
\begin{description}
\item[(5)] 本Git勉強会では,Gitの概要と,その仕組み,および使い方について学んだ.これまでほとんどGitを利用することがなかったが,これを機に積極的にGitを利用し,Gitを十分に使いこなせるようになる.
\end{description}
\subsection{大学関連}
特になし.

\section{今後の予定}
\subsection{研究関連}
\begin{enumerate}
\item (4/27) 第351回New打ち合わせ
\end{enumerate}
\subsection{研究室関連}
\begin{enumerate}
\item (5/14) 乃村研ミーティング
\item (5/11) Patchwork Okayama -2018
\end{enumerate}
\subsection{大学関連}
\begin{enumerate}
\item (5/14,1521)非手続き型言語 
\end{enumerate}

\bibliographystyle{ipsjunsrt}
%\bibliography{mybibdata}
%\begin{thebibliography}{}
%\end{thebibliography}
\end{document}
