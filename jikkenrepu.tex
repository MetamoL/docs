\documentclass[11pt,a4j]{jarticle}
\usepackage{comment}

\title{情報工学実験A \\ 
       レポート
}
\author{吉田修太郎 \and 09425562}
\date{提出日:2017/05/26 \\
      締切日:2017/05/29
}
\begin{document}
\maketitle
\section{はじめに}
このレポートでは今回の実験内容に関する報告をまとめる.今回の実験はSFLによるプロセッサ設計を通じてCPUの構造およびその動作について理解を深めることを目的とするものである.まずはじめに32bitプロセッサの設計課題に関して報告する.ここでは設計したプロセッサに共通する基本構成および各サブモジュールに関する説明も掲載する.次にプログラミング課題について報告する.ここで採り上げるプログラムは,設計したプロセッサの上で実際に実行するためのものである.次章では設計した3つのプロセッサごとの諸元と,前章で採り上げたプログラムの実行結果およびその考察等を述べる.次に追加課題として設計したCLAに関する報告を行う.次章では作成したCLAを設計したプロセッサに実際に組み込んで動作させる.最後の章はこの実験の総括と所感を述べる.
\section{プロセッサ構成}
ここでは今回設計した32bitプロセッサについて簡単に説明する.
設計したのは1命令あたりのサイクル数がすべて5サイクルのものと命令毎に異なる(2~5サイクル)もの,パイプライン方式を採用したものの3種類である.それぞれ呼称をp32m1,p32m2,p32p1とする.命令セットはMIPS命令セット.ユニットを構成するサブモジュールの設計から合成までが,今回の課題である.設計にあたってはハードウェア言語SFLで記述しPARTHENONで論理合成を行った.動作確認にはシミュレータSECONDSを用いた.プロセッサの基本構成は以下の通り.\\ \newpage
{\scriptsize
 プロセッサの基本構成
\begin{verbatim}
top_p32
 |
 +- p32
 |   |
 |   +- p32decode_unit
 |   |   |
 |   |   +- add32
 |   |       |
 |   |       +- fulladder
 |   |
 |   +- p32exec_unit
 |   |   |
 |   |   +- alu32
 |   |   |   |
 |   |   |   +- add32
 |   |   |       |
 |   |   |       +- fulladder
 |   |   |
 |   |   +- shift32
 |   |
 |   +- reg32x32
 |   |   |
 |   |   +- reg32x8
 |   |
 |   +- add32
 |       |
 |       +- fulladder
 |
 +- memunit
     |
     +- mem64KB
\end{verbatim}
}
\subsection{サブモジュールの設計について}
ここでは先に示したモジュールの内,単一で論理合成と動作確認を行ったものについてその結果を示す.
\subsubsection{add32}
32bit加算器.fulladder(全加算器)を32個つなげた構造になっている.32bit幅の入力2つと1bitのキャリーの和を32bit信号と1bitのキャリーで出力する.
\begin{table}[htb]
\caption{add32:諸元}
\begin{tabular}{|c|c|c|c|} \hline 
最大遅延(ns) & 最大遅延(比) & 最大動作周波数(MHz) & 等価ゲート数 \\ \hline
117.9 & (1.09) & 8.5 & 561 \\ \hline \hline
面積(1000$\mu$m$^2$) & 消費電力($\mu$W/MHz) & 最大消費電力(mW) & \\ \hline 
147.66 & 1407.7 & 11.94 & \\ \hline 
\end{tabular}
\end{table}
\subsubsection{alu32}
32bit演算装置.各種演算を行う.加算減算はサブモジュールのadd32で行い,ほか論理演算などはそれぞれに対応した回路で行う.乗除算専用の回路は持っていない.2つの32bit入力に対する演算結果を32bitで出力する.
\begin{table}[htb]
\caption{alu32:諸元}
\begin{tabular}{|c|c|c|c|} \hline
 最大遅延(ns) & 最大遅延(比) & 最大動作周波数(MHz) & 等価ゲート数 \\ \hline
23.6 & (0.24) & 42.4 & 1647 \\ \hline \hline
面積(1000$\mu$m$^2$) & 消費電力($\mu$W/MHz) & 最大消費電力(mW) & \\ \hline
430.04 & 3451.1 & 146.23 & \\ \hline
\end{tabular}
\end{table}
\subsubsection{shift32}
32bitシフタ.シフト演算を行う.32bitの入力信号に対して5bitのシフト量入力信号に基づいたシフト操作を行い,結果を32bitで出力する.
\begin{table}[htb]
\caption{shift32:諸元}
\begin{tabular}{|c|c|c|c|} \hline
最大遅延(ns) & 最大遅延(比) & 最大動作周波数(MHz) & 等価ゲート数 \\ \hline
36.6 & (1.00) & 27.3 & 1249 \\ \hline \hline
面積(1000$\mu$m$^2$) & 消費電力($\mu$W/MHz) & 最大消費電力(mW) & \\ \hline
311.13 & 2706.5 & 73.95 & \\ \hline
\end{tabular}
\end{table}
\subsubsection{reg32x32}
32bit幅x32のレジスタ.8bit幅x32のレジスタ4個から成る.5bitのアドレス入力信号によって指定されレジスタに32bit幅の入力信号を書き込む.あるいは指定のレジスタの内容を32bit信号で出力する.
\begin{table}[htb]
\caption{reg32x32:諸元}
\begin{tabular}{|c|c|c|c|} \hline
最大遅延(ns) & 最大遅延(比) & 最大動作周波数(MHz) & 等価ゲート数 \\ \hline
26.3 & (0.82) & 38.0 & 18298 \\ \hline \hline 
面積(1000$\mu$m$^2$) & 消費電力($\mu$W/MHz) & 最大消費電力(mW) & \\ \hline
4964.62 & 36394.0 & 1383.80 & \\ \hline
\end{tabular}
\end{table}

\section{プログラミング課題}
ここでは今回の実験で作成したMIPSアセンブリコードによるソートプログラムについて報告する.このプログラムは設計したプロセッサのテストに用いるためのものである.手法はバブルソートを採用した.以下諸元

\begin{table}[htb]
\caption{諸元}
\begin{tabular}{|c|c|c|c|c|} \hline
データ数 & 総命令数 & 総サイクル数 & IPC & CPI \\ \hline
512 & 2097473 & 7733370 & 0.271 & 3.687 \\ \hline
128 & 128313 & 473438 & 02.71 & 3690 \\ \hline
32 & 8401 & 30898 & 02.72 & 3678 \\ \hline
\end{tabular}
\end{table}
%このソートプログラムを用いて,次章ではプロセッサの性能比較を行う.
\begin{verbatim}
\end{verbatim}

\section{プロセッサ設計課題}
ここでは設計した各プロセッサごとの諸元
%およびソートプログラムの実行結果
について報告する.

\subsection{マルチサイクルモデル1}
全ての命令が5サイクルで実行されるマルチサイクルモデル.命令フェッチ,命令デコード,実行,メモリアクセス,ライトバックの5ステージを経て1命令実行となる.
\begin{table}[htb]
\caption{諸元}
\begin{tabular}{|c|c|c|c|} \hline
最大遅延(ns) & 最大遅延(比) & 最大動作周波数(MHz) & 等価ゲート数 \\ \hline
101.8 & (1.04) & 9.8 & 28284 \\ \hline \hline
面積(1000$\mu$m$^2$) & 消費電力($\mu$W/MHz) & 最大消費電力(mW) & \\ \hline
7622.07 & 56935.1 & 559.28 & \\ \hline
\end{tabular}
\end{table}

%\begin{table}[htb]
%\caption{ソートプログラム実行結果}
%\begin{tabular}{|c||c|c|c|} \hline
%\データ数 & 32 & 128 & 512 \\ \hline \hline
%実行サイクル数 & 84 & 2030 & 0x204c \\ \hline
%実行時間 &  &  &  \\ \hline
%消費エネルギー &  &  &  \\ \hline
%エネルギー時間積 &  &  &  \\ \hline
%\end{tabular}
%\end{table}

\subsection{マルチサイクルモデル2}
命令によって実行に必要なサイクル数が変わる(2~5サイクル)モデル.マルチサイクルモデル1を基本に,特定の命令を実行する場合に不要なステージをスキップする処理を付け加えている.
\begin{table}[htb]
\caption{諸元}
\begin{tabular}{|c|c|c|c|} \hline
最大遅延(ns) & 最大遅延(比) & 最大動作周波数(MHz) & 等価ゲート数 \\ \hline
98.3 & (1.10) & 10.2 & 28652 \\ \hline \hline
面積(1000$\mu$m$^2$) & 消費電力($\mu$W/MHz) & 最大消費電力(mW) & \\ \hline
7712.56 & 57609.5 & 586.06 & \\ \hline
\end{tabular}
\end{table}

\subsection{パイプラインモデル}
パイプライン処理を実装したモデル.5つあるステージそれぞれに対応したモジュール(回路)を遊ばせない為に,全てのステージで常に何らかのジョブの処理が行われる.全ての命令が5サイクルかけて実行されるが,理論上は単純計算でマルチサイクルモデルの5倍のスループットになる.ただし,分岐遅延の問題もあるため実際はそこまでの効率は得られない.また,データフォワーディングのためのモジュールが別途追加されている.
\begin{table}[htb]
\caption{諸元}
\begin{tabular}{|c|c|c|c|} \hline
最大遅延(ns) & 最大遅延(比) & 最大動作周波数(MHz) & 等価ゲート数 \\ \hline
126.0 & (1.13) & 7.9 & 31102 \\ \hline \hline
面積(1000$\mu$m$^2$) & 消費電力($\mu$W/MHz) & 最大消費電力(mW) & \\ \hline
8385.59 & 63304.5 & 502.42 & \\ \hline
\end{tabular}
\end{table}

\subsection{性能比較}
結果を比べるとわずかではあるがマルチサイクルモデル1よりもマルチサイクルモデル2の方が高速動作する.命令によっては処理にかかるサイクル数が少なくなることも考えると実際の差は広がる.そしてパイプラインモデルが突出して処理能力が高い.最大動作周波数を比べると他2つのモデルに劣っているものの,理論上は1サイクルあたり1命令を実行している事になる為である.これを加味すればマルチサイクルモデル1の約4倍の処理能力を持っているといえる.
\section{追加課題}
\subsection{cla32}
ここでは指定課題のほかに,追加で設計したモジュールについて述べる.作成したのは4bitキャリ先読み加算器を8つ用いた32bitキャリ予測加算器である.はじめは32bitのキャリを全て並列に演算するモジュールを考えていたが,上位ビットのキャリほど論理式が膨大になるためかシュミレーションの際実習用PCのスタック溢れを引き起こしたため諦めることとなった.そこでキャリの先読みは4ビットにとどめ,それを直列に8個つなげる構成にした.ただ,そのままつなぎ合わせたのではキャリ先読みの恩恵がほとんど無いため,つなぎ方に工夫をした.具体的には,下位ビットの演算結果を待たずにキャリーが1の時と0の時との両方の演算を行っておき,後から判明したキャリの値に合った演算結果を選ぶという方式である.なのでこれを32bitCLAと呼称するのはいささか憚られるのだが,モジュール名は便宜的にcla32としている.以下諸元.\\
\begin{table}[htb]
\caption{cla32:諸元}
\begin{tabular}{|c|c|c|c|} \hline
最大遅延(ns) & 最大遅延(比) & 最大動作周波数(MHz) & 等価ゲート数 \\ \hline
61.6 & (1.22) & 16.2 & 1434 \\ \hline \hline
面積(1000$\mu$m$^2$) & 消費電力($\mu$W/MHz) & 最大消費電力(mW) & \\ \hline
371.55 & 3088.6 & 50.14 & \\ \hline
\end{tabular}
\end{table}
通常の32bit加算器に比べて遅延が約半分,ゲート数が2倍ほどになった.下位4ビット以外は並列演算をしているのでゲート数が約2倍になるのは妥当である.さらに遅延を少なくするためには,サブモジュールとして組み込むキャリ先読み加算器のビット幅を増やすことが必要となる.全ビットのキャリを先読みする事が出来れば,それがもっとも遅延が少なくなると考えられる.
\section{検討・考察}
p32m1に追加で作成したcla32をadd32と差し替える形で組み込んで動作させた.呼称はp32m1rとする.論理合成結果は表10に示す.\\
\begin{table}[htb]
\caption{p32m1r:諸元}
\begin{tabular}{|c|c|c|c|} \hline
最大遅延(ns) & 最大遅延(比) & 最大動作周波数(MHz) & 等価ゲート数 \\ \hline
73.5 & (0.83) & 13.6 & 29375 \\ \hline \hline
面積(1000$\mu$m$^2$) & 消費電力($\mu$W/MHz) & 最大消費電力(mW) & \\ \hline
7891.50 & 59532.2 & 809.96 & \\ \hline
\end{tabular}
\end{table}
表をみれば,組み込み前と比べ順当に遅延が少なくなり高速化していることが分かる.遅延の減少の逆数分だけ最大動作周波数が上昇している事が伺える.そして最大消費電力が最大動作周波数の上昇分以上の割合で増加していることが分かる.ここからさらにプロセッサを高速化する手段として,キャッシュレジスタの搭載が考えられる.
\section{おわりに}
今回の実験では3種類のプロセッサを設計し,その性能を比較した.また性能比較には論理合成の他に自作したソートプログラムも用いた.さらにキャリ先読み加算器等のモジュールを追加で設計し,プロセッサに組み込むことも行った.
\appendix
{\Large \gt 付録}
\begin{table}[htb]
\caption{追加課題設計状況一覧}
\begin{center}
\begin{tabular}{rll|l}
\hline
\hline
\multicolumn{3}{c|}{モジュール} & 設計状況 \\
\hline
1. & 4ビットCLA        & \verb|cla4|          & 動作確認済 \\
2. & 32ビットキャリー予測加算器   & \verb|cla32|         & 論理合成済 \\
\hline
\end{tabular}
\end{center}
\end{table}
\begin{table}[htb]
\caption{GitBucketのリポジトリURL}
\begin{center}
\begin{tabular}{|l|l|l|}
\hline
モジュール & リポジトリ & URL \\
\hline\hline
p32m1 & p31m1 & http://jikken1.arc.cs.okayama-u.ac.jp/gitbucket/09425562/p32m1 \\ \hline
p32m2 & p32m2 & http://jikken1.arc.cs.okayama-u.ac.jp/gitbucket/09425562/p32m2 \\ \hline
p32p1 & p32p1 & http://jikken1.arc.cs.okayama-u.ac.jp/gitbucket/09425562/p32p1 \\ \hline
p32m1r & p32m1r & http://jikken1.arc.cs.okayama-u.ac.jp/gitbucket/09425562/p32m1\verb|_|r \\ \hline
\end{tabular}
\end{center}
\end{table}
\end{document}
