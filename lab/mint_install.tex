\documentclass[12pt]{jsarticle}
\usepackage[dvipdfmx]{graphicx}
\textheight = 25truecm
\textwidth = 18truecm
\topmargin = -1.5truecm
\oddsidemargin = -1truecm
\evensidemargin = -1truecm
\marginparwidth = -1truecm

\def\theenumii{\Alph{enumii}}
\def\theenumiii{\alph{enumiii}}
\def\labelenumi{(\theenumi)}
\def\labelenumiii{(\theenumiii)}
\def\theenumiv{\roman{enumiv}}
\def\labelenumiv{(\theenumiv)}
\usepackage{comment}

%%%%%%%%%%%%%%%%%%%%%%%%%%%%%%%%%%%%%%%%%%%%%%%%%%%%%%%%%%%%%%%%
%% sty/ にある研究室独自のスタイルファイル
\usepackage{jtygm}  % フォントに関する余計な警告を消す
\usepackage{nutils} % insertfigure, figref, tabref マクロ

\def\figdir{./figs} % 図のディレクトリ
\def\figext{pdf}    % 図のファイルの拡張子

\begin{document}
%%%%%%%%%%%%%%%%%%%%%%%%%%%%
%% 表題
%%%%%%%%%%%%%%%%%%%%%%%%%%%%
\begin{center}
  {\LARGE Linuxカーネルへのシステムコール追加の手順書}
\end{center}

\begin{flushright}
  2018/4/20\\
  吉田 修太郎
\end{flushright}
%%%%%%%%%%%%%%%%%%%%%%%%%%%%
%% 概要
%%%%%%%%%%%%%%%%%%%%%%%%%%%%
\section{はじめに}
%本手順書の章立てを以下に示す.
%\begin{description}
%\item[第1章] はじめに
%\item[第2章] Mintの環境構築手順
%\item[第3章] システムコールの追加手順
%%%%%%%%%%%%%%%%%%%%%%%%%%%%%%%%%%%%%%%%%%%%%%%%%%%%%
%\item[第4章] システムコールの動作テスト
%\item[第5章] おわりに
%\end{description}
\section{システムコールを追加する実装環境}\label{sec2}
本手順書においてシステムコールを追加する実装環境を表\ref{table1}に示す.
\begin{table}[h!]
  \begin{center}
    \caption{実装環境}%\label{tab:time_range_ratio}
    %\ecaption{Frequency of task ocurrences.}
    \begin{tabular}{l|l}
      \hline\hline
      OS & Debian 7.11 \\
      カーネル & Linux カーネル 3.2.0-4-amd64 \\
      マザーボード & H87-PLUS (ASUSTeK COMPUTER INC.)\\
      CPU & Intel(R) Core(TM) i7-4770 \\
      メモリ & 16GB\\
      \hline
    \end{tabular}
    \label{table1}
  \end{center}
\end{table}


\begin{description}
\item[] カーネルの再構築に必要なライブラリとコマンドをインストールする.以下のコマンドを実行する.
  \begin{description}
  \item[] \verb|$ sudo apt-get update|
  \item[] \verb|$ sudo apt-get install git|
  \item[] \verb|$ sudo apt-get install gcc|
  \item[] \verb|$ sudo apt-get install bc |
  \item[] \verb|$ sudo apt-get install make|
  \end{description}
\item[] 実行により,\verb|git|,\verb|gcc|,\verb|bc|,および\verb|make|をインストールする.これらは,以降の操作で用いるコマンドの実行に必要となる.
\end{description}



上記スクリプトの内容に関する補足を以下に示す.
\begin{enumerate}
\item \verb|mementry <表示名>|
  \begin{description}
  \item \verb|<表示名>|: ブートローダのカーネル選択画面に表示される名前.本手順書ではMy custom Linuxとしている.
  \end{description}
\item \verb|set root=(<HDD番号>,<パーティション番号>)|
  \begin{description}
  \item \verb|<HDD番号>|: カーネルが保存されているHDDの番号
  \item \verb|<パーティション番号>|: HDDの\verb|/boot|が割り当てられたパーティション番号
  \end{description}
\item \verb|ro <rootデバイス>|
  \begin{description}
  \item \verb|<rootデバイス>|: 起動時に読み込み専用でマウントするデバイス
  \end{description}
\item \verb|linux <カーネルイメージのファイル名>|
  \begin{description}
  \item \verb|<カーネルイメージのファイル名>|: 起動するカーネルのカーネルイメージ
  \end{description}
\item \verb|root=<ルートファイルシステム> <その他のブートオプション>|
  \begin{description}
  \item \verb|<ルートファイルシステム>|: \verb|/root|を割り当てたパーティション
  \item \verb|<その他のブートオプション>|: \verb|quiet|はカーネル起動時に出力するメッセージを省略する.
  \end{description}
\item \verb|initrd <初期RAMディスク名>|
  \begin{description}
  \item \verb|<初期RAMディスク名>|: 起動時にマウントする初期RAMディスク名
  \end{description}
\end{enumerate}

\section{おわりに}
本手順書では,Linuxカーネルへ新たにシステムコールを追加する手順について述べた.さらに,追加したシステムコールが,意図した動作をしているかどうかを調べるテストの手順について述べた.

\bibliographystyle{ipsjunsrt}
\bibliography{mybibdata}
%\begin{thebibliography}{1}
%\bibitem{jobs}
%\end{thebibliography}
\end{document}
