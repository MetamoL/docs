\documentclass[12pt]{jsarticle}
\usepackage[dvipdfmx]{graphicx}
\textheight = 25truecm
\textwidth = 18truecm
\topmargin = -1.5truecm
\oddsidemargin = -1truecm
\evensidemargin = -1truecm
\marginparwidth = -1truecm

\def\theenumii{\Alph{enumii}}
\def\theenumiii{\alph{enumiii}}
\def\labelenumi{(\theenumi)}
\def\labelenumiii{(\theenumiii)}
\def\theenumiv{\roman{enumiv}}
\def\labelenumiv{(\theenumiv)}
\usepackage{comment}

%%%%%%%%%%%%%%%%%%%%%%%%%%%%%%%%%%%%%%%%%%%%%%%%%%%%%%%%%%%%%%%%
%% sty/ にある研究室独自のスタイルファイル
\usepackage{jtygm}  % フォントに関する余計な警告を消す
\usepackage{nutils} % insertfigure, figref, tabref マクロ

\def\figdir{./figs} % 図のディレクトリ
\def\figext{pdf}    % 図のファイルの拡張子

\begin{document}
%%%%%%%%%%%%%%%%%%%%%%%%%%%%
%% 表題
%%%%%%%%%%%%%%%%%%%%%%%%%%%%
\begin{center}
{\LARGE SlackBotプログラム仕様書}
\end{center}

\begin{flushright}
  2018/4/26\\
  吉田 修太郎
\end{flushright}
%%%%%%%%%%%%%%%%%%%%%%%%%%%%
%% 概要
%%%%%%%%%%%%%%%%%%%%%%%%%%%%
\section{概要}
\label{sec:introduction}
%\section{概要} % ※400~1,200文字程度
% - レポートの概要を述べる
\section{課題内容}

\subsection{SlackBotプログラムとは}
本課題は,SlackBotプログラムを作成するというものである.SlackBotプログラムとは,Slackの投稿を契機として,定められた処理を自動で行うプログラムである.ただし,任意に追加される機能については,必ずしもSlackの投稿を契機とするわけではない.
\subsection{作成するSlackBotプログラムの要件}
以下の2つの機能を持つSlackBotプログラムをRubyで作成する.
\begin{enumerate}
\item 任意の文字列を投稿するプログラムの作成\\
  受信した投稿の中に``「○○○」と言って''という文字列があった場合は``○○○''と投稿する.
\item SlackBotプログラムへの機能追加\\
  SlackBotプログラムへ機能を追加する.Slack以外のWebサービスのAPIやWebhookを利用した機能を追加する.
\end{enumerate}
\section{自主的に実装した機能}
以下の機能を実装した.
\begin{enumerate}
\item 
\item SlackBotプログラムへの機能追加\\
\end{enumerate}
\section{実装できなかった機能}
以下の機能については,必要性を感じつつも,実装することができなかった.
\begin{enumerate}
\item 
\item SlackBotプログラムへの機能追加\\
\end{enumerate}
\section{理解できなかった部分}
%%%%%%%%%%%%%%%%%%%%%%%%%%%%%%%%%%%%%%%%%%%%%%%%%%%%%%%%%%%%%%%%%%%%%%

%%%%%%%%%%%%%%%%%%%%%%%%%%%%%%%%%%%%%%%%%%%%%%%%%%%%%%%%%%%%%%%%%%%%%%
% 参考文献リスト

\begin{thebibliography}{1}
  \bibitem{bootcamp} 乃村研究室 資料 https:\slash{}\slash{}github.com\slash{}nomlab\slash{}BootCamp\slash{}blob\slash{}master\slash{}2018\slash{}README.org
\end{thebibliography}

%\bibliographystyle{ipsjunsrt}
%\bibliography{mybibdata}

\end{document}
