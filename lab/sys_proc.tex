\documentclass[12pt]{jsarticle}
\usepackage[dvipdfmx]{graphicx}
\textheight = 25truecm
\textwidth = 18truecm
\topmargin = -1.5truecm
\oddsidemargin = -1truecm
\evensidemargin = -1truecm
\marginparwidth = -1truecm

\def\theenumii{\Alph{enumii}}
\def\theenumiii{\alph{enumiii}}
\def\labelenumi{(\theenumi)}
\def\labelenumiii{(\theenumiii)}
\def\theenumiv{\roman{enumiv}}
\def\labelenumiv{(\theenumiv)}
\usepackage{comment}

%%%%%%%%%%%%%%%%%%%%%%%%%%%%%%%%%%%%%%%%%%%%%%%%%%%%%%%%%%%%%%%%
%% sty/ にある研究室独自のスタイルファイル
\usepackage{jtygm}  % フォントに関する余計な警告を消す
\usepackage{nutils} % insertfigure, figref, tabref マクロ

\def\figdir{./figs} % 図のディレクトリ
\def\figext{pdf}    % 図のファイルの拡張子

\begin{document}
%%%%%%%%%%%%%%%%%%%%%%%%%%%%
%% 表題
%%%%%%%%%%%%%%%%%%%%%%%%%%%%
\begin{center}
{\LARGE 資料タイトル}
\end{center}

\begin{flushright}
  2018/4/10\\
  吉田 修太郎
\end{flushright}
%%%%%%%%%%%%%%%%%%%%%%%%%%%%
%% 概要
%%%%%%%%%%%%%%%%%%%%%%%%%%%%
\section{はじめに}
本手順書では,Linuxカーネルに対して新たにシステムコールを追加するための手順について述べる.以降では,実装環境,実装したシステムコールの概要,実装の手順,動作テストについて,順に章立ててそれぞれの詳細を述べる.

\section{実装環境}
本手順書における実装環境を下表に示す.
\begin{table}[tb]
  \begin{center}
    \caption{実装環境}%\label{tab:time_range_ratio}
    %\ecaption{Frequency of task ocurrences.}
    \begin{tabular}{r|r}
      \hline\hline
      OS & Debian 7.11 \\
      \hline
      カーネル & Linux カーネル 3.15.0\\
      \hline
      CPU & Intel(R) Core(TM) i7-4770
      \hline
      メモリ & 
      \hline
    \end{tabular}
  \end{center}
\end{table}

\section{実装するシステムコールの概要}
ここでは,本手順書において実装するシステムコールの概要について述べる.
\begin{description}
\item[形式] asmlinkage int prt\_to\_rbuf(char *s)
\item[引数] char *s: 出力する文字列のポインタ
\item[戻り値] カーネルバッファに書き込んだ文字数
  %\begin{description}
  %\item[成功]:カーネルバッファに書き込んだ文字数
  %\item[失敗]:-1
  %\end{description}
\item[機能] 受け取った文字列をカーネルバッファに書き込む
  
\end{description}
\section{実装の手順}
\subsection{概要}
本章では,システムコール実装の手順について述べる.以降では,ソースコードの作成,プロトタイプ宣言,システムコール番号の定義,Makefileの編集およびカーネルの再構築についてそれぞれについて節を設けて詳細に述べる.
\subsection{ソースコードの作成}

\subsection{プロトタイプ宣言}
ここでは,プロトタイプ宣言の手順について述べる.システムコール関数のプロトタイプ宣言が,まとめて書かれているヘッダファイルを探し,編集する.本手順書では,以下のファイルを編集する.
\begin{itemize}
\item \slash{}home\slash{}git\slash{}linux-stable\slash{}include\slash{}linux\slash{}syscalls.h
\end{itemize}
本手順書では,このヘッダファイルの末尾に以下の行を追加する.
\begin{itemize}
\item \slash{}home\slash{}git\slash{}asmlinkage int prt\_to\_rbuf(char *s);
\end{itemize}
\subsection{システムコール番号の定義}
ここでは,システムコール番号を定義する手順について述べる.システムコール関数と,システムコール番号との対応づけが書かれているファイルを探し,編集する.本手順書では,以下のファイルを編集する.
\begin{itemize}
\item \slash{}home\slash{}git\slash{}linux-stable\slash{}arch\slash{}x86\slash{}syscalls\slash{}syscall\_64.tbl
\end{itemize}
このファイルの内容の一部を抜粋し,以下に示す.
\begin{verbatim}
\end{verbatim}
上記のファイル内容の先頭付近に,このファイルのフォーマットは<number><abi><name><entry point>であるという旨が記述されている.それぞれの要素についての簡単な説明を以下に示す.
\begin{description}
\item[number] システムコール番号
\item[abi] Applicatioin Binary Interface
\item[name] 関数名
\item[entry point] 関数が 
\end{description}
このフォーマットに従い,実装したいシステムコールをこのファイルに追記する.ただし,システムコール番号は,システムコール呼出しの際に関数の特定に使用されるため,他の関数と重複してはならない.本資料では,以下のように設定する.
\begin{description}
\item[number] 317
\item[abi] common
\item[name] sys\_prt\_to\_rbuf
\item[entry point] sys\_prt\_to\_rbuf 
\end{description}
上記の場合におけるファイルへの記入例を以下に示す.
\begin{verbatim}
\end{verbatim}
\subsection{Makefile編集}
ここでは,Makefileの編集について述べる.今回編集するMakefileを以下に示す.
\begin{itemize}
\item \slash{}home\slash{}git\slash{}linux-stable\slash{}kernel\slash{}Makefile
\end{itemize}
makeコマンドは,このファイルの内容に基づいて実行されるため,今回追加したシステムコール関数をコンパイルするためには,ここにその処理を追記する必要がある.具体的には,各システムコール関数のオブジェクトファイルが代入されるobj-yという変数に対して,新たに作成したシステムコールのオブジェクトファイルも代入されるように追記する.本手順書における,このMakefileの編集内容を以下に示す.

\begin{verbatim}
変更前
\end{verbatim}

\begin{verbatim}
変更後
\end{verbatim}
\subsection{カーネルの再構築}

\section{動作テスト}
\subsection{概要}
本章では,実装したシステムコールの動作テストについて述べる.
\subsection{動作テスト用プログラムの準備}
prt\_to\_rbufの動作テストに用いるプログラムを作成する.本手順書における動作テスト用プログラムを以下に示す.
\subsection{動作テストの手順}
動作テストの手順を以下に示す.
\begin{enumerate}
\item 動作テスト用プログラムの実行
\item dmesgコマンドの実行
\end{enumerate}
\subsection{参考文献の挿入例}
参考文献を記載する際はbibtexを利用する.
mybibdate.bibに参考文献の情報を記載する.
たとえば,乃村先生の論文\cite{nom2011c}を参考文献として記載する.

\section{おわりに}

本資料では

\bibliographystyle{ipsjunsrt}
\bibliography{mybibdata}

\end{document}
