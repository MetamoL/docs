\documentclass[12pt]{jsarticle}
\usepackage[dvipdfmx]{graphicx}
\textheight = 25truecm
\textwidth = 18truecm
\topmargin = -1.5truecm
\oddsidemargin = -1truecm
\evensidemargin = -1truecm
\marginparwidth = -1truecm

\def\theenumii{\Alph{enumii}}
\def\theenumiii{\alph{enumiii}}
\def\labelenumi{(\theenumi)}
\def\labelenumiii{(\theenumiii)}
\def\theenumiv{\roman{enumiv}}
\def\labelenumiv{(\theenumiv)}
\usepackage{comment}

%%%%%%%%%%%%%%%%%%%%%%%%%%%%%%%%%%%%%%%%%%%%%%%%%%%%%%%%%%%%%%%%
%% sty/ にある研究室独自のスタイルファイル
\usepackage{jtygm}  % フォントに関する余計な警告を消す
\usepackage{nutils} % insertfigure, figref, tabref マクロ

\def\figdir{./figs} % 図のディレクトリ
\def\figext{pdf}    % 図のファイルの拡張子

\begin{document}
%%%%%%%%%%%%%%%%%%%%%%%%%%%%
%% 表題
%%%%%%%%%%%%%%%%%%%%%%%%%%%%
\begin{center}
  {\LARGE Linuxカーネルへのシステムコール追加の手順書}
\end{center}

\begin{flushright}
  2018/4/16\\
  吉田 修太郎
\end{flushright}
%%%%%%%%%%%%%%%%%%%%%%%%%%%%
%% 概要
%%%%%%%%%%%%%%%%%%%%%%%%%%%%
\section{はじめに}
本手順書では,Linuxカーネルに対して新たにシステムコールを追加するための手順について述べる.本手順書で追加するシステムコールは,任意の文字列をカーネルのメッセージバッファに書き込む機能を持つシステムコールである.なお,本手順書はコンソールの基本操作と\verb|Git|の基礎知識を習得している者を対象としている.
%本手順書において示される一連の操作の中で,必要となるあらゆるパッケージ(詳細は第\ref{sec2}章に記載)はすでにインストールされているものとする.
以降では,追加するシステムコールの実装環境,追加するシステムコールの概要,システムコールの追加手順,システムコールの動作テストについて,順に章を設けて説明する.

\section{追加するシステムコールの実装環境}\label{sec2}
本手順書において追加するシステムコールの実装環境を表\ref{table1}に示す.
\begin{table}[h!]
  \begin{center}
    \caption{実装環境}%\label{tab:time_range_ratio}
    %\ecaption{Frequency of task ocurrences.}
    \begin{tabular}{l|l}
      \hline\hline
      OS & Debian 7.11 \\
      \hline
      カーネル & Linux カーネル 3.15.0 \\
      \hline
      CPU & Intel(R) Core(TM) i7-4770 \\
      \hline
      メモリ & 16GB\\
      \hline
    \end{tabular}
    \label{table1}
  \end{center}
\end{table}

\section{システムコールの追加手順}
\subsection{概要}
本章では,新たにシステムコールを追加する手順について述べる.以降では,環境設定,Linuxカーネルの入手,ソースコードの作成,関数のプロトタイプ宣言,システムコール番号の定義,Makefileの編集,およびカーネルの再構築についてそれぞれについて節を設けて説明する.また,本章においてファイルの内容を編集する際には,追加した行の先頭に\verb|+|を,削除した行の先頭に\verb|-|を記して示す.なお,以降で現れる\verb|username|は,操作を行う計算機にログインしているユーザ名である.
\subsection{環境設定}
\begin{enumerate}
\item \verb|sudo|権限の付与
  \begin{description}
  \item[] 以降の作業を行うユーザに\verb|sudo|権限を付与する.以下のコマンドを実行する.
    \begin{description}
    \item[] \verb|$ su|
    \item[] \verb|# visudo|
    \end{description}
  \item[] 実行後エディタが起動し,\verb|sudoers.tmp|が編集できるようになる.ここで,\verb|sudoers.tmp|を以下のように編集する.
\begin{verbatim}
 19 # User privilege specification
 20 root    ALL=(ALL:ALL) ALL
+21 username ALL=(ALL) ALL
\end{verbatim}
    \item[] 編集後,\verb|username|に\verb|sudo|権限が付与される.
  \end{description}
\item \verb|git|と\verb|gcc|のインストール
  \begin{description}
  \item[] gitとgccをインストールする.以下のコマンドを実行する.
    \begin{description}
    \item[] \verb|$ sudo apt-get update|
    \item[] \verb|$ sudo apt-get install git|
    \item[] \verb|$ sudo apt-get install gcc|
    \end{description}
    \item[] 実行後,\verb|git|と\verb|gcc|がインストールされる.
  \end{description}
\item ライブラリとコマンドのインストール
  \begin{description}
  \item[] カーネル再構築に必要なライブラリとコマンドをインストールする.以下のコマンドを実行する.
    \begin{description}
    \item[] \verb|$ sudo apt-get install libncurses5-dev|
    \item[] \verb|$ sudo apt-get install bc |
    \end{description}
  \item[] 実行後,\verb|libncurses5-dev|と\verb|bc|がインストールされる.これらは,カーネルを構築する際に用いるコマンドの実行に必要となる.
  \end{description}
\end{enumerate}

\subsection{Linuxカーネルの入手}
\begin{enumerate}
\item ソースコードの入手
  \begin{description}
  \item[] Linuxのソースコードを入手する.Linuxのソースコードが管理されている\verb|Git|のリポジトリをクローンする.本手順書では,\verb|/home/username|以下に\verb|git|ディレクトリを作成し,\verb|/home/username/git|以下でクローンする.以下のコマンドを実行する.
    \begin{description}
    \item[] \verb|$ cd /home/username|
    \item[] \verb|$ mkdir git|
    \item[] \verb|$ cd git|
    \item[] \verb|$ git clone git://git.kernel.org/pub/scm/linux/kernel|
    \item[] \verb|  /git/stable/linux-stable.git|
    \end{description}
  \item[] 実行後,\verb|/home/username/git|以下に\verb|linux-stable|ディレクトリが作成され,\verb|/home/username/git/linux-stable|以下に,Linuxのソースコードが格納される.
  \end{description}
\item ブランチの作成と切替え
  \begin{description}
  \item[] ブランチの作成と切替えを行う.本手順書で使用するLinuxのバージョンは3.15であるため,そのバージョンのコミットへ移動する.そこで,新規のブランチを作成した後,新規のブランチへカレントブランチを変更する.以降で示される操作は,すべて\verb|/home/username/git/linux-stable|以下で行う.以下のコマンドを実行する.
    \begin{description}
    \item[] \verb|$ git checkout -b 3.15 v3.15|
    \end{description}
  \item[] 実行後,v3.15というタグが示すコミットから新規のブランチ3.15が作成され,カレントブランチがブランチ3.15に切り替わる.
  \end{description}
\end{enumerate}

\subsection{ソースコードの作成}
追加するシステムコールのソースコードを作成する.本手順書で追加するシステムコールの概要を以下に示す.
\newpage
\begin{description}
  \item[形式] :asmlinkage int sys\_prt\_to\_rbuf(char *s)
  \item[引数] :char *s: 任意の文字列を指すポインタ
  \item[戻り値] :カーネルのメッセージバッファに書き込んだ文字数
  \item[機能] :任意の文字列をカーネルのメッセージバッファに書き込み,\\
      さらに"(number) charactor(s) outputed"という文字列を\\
      カーネルのメッセージバッファに書き込む.(number)の部分は,\\
      カーネルのメッセージバッファに書き込んだ任意の文字列の文字数である.\\
\end{description}

%\item[(B)] 追加するシステムコールのソースコード
%  \begin{description}
%    \item[]
%\begin{verbatim}
%1 #include <linux/kernel.h>
%2 #include <linux/syscalls.h>
%3
%4 asmlinkage int sys_prt_to_rbuf(char *s){
%5  int ret;
%6  ret = printk(KERN_INFO "%s\n",s);
%7  printk("%d charactor(s) outputed\n",ret);
%8  return ret;
%9}
%\end{verbatim}
%    \item[] 上記コードでは,任意の文字列をカーネルのメッセージバッファに書き込むために,6行目でprintkを呼び出している.printk printtorbuf.cです
%  \end{description}
%\end{description}
\subsection{関数のプロトタイプ宣言}
関数のプロトタイプ宣言を行う.
本手順書では,\verb|/home/username/git/linux-stable/include/|\\
\verb|linux/syscalls.h|を以下のように編集する.
\begin{verbatim}
 866 asmlinkage long sys_kcmp(pid_t pid1, pid_t pid2, int type,
 867                          unsigned long idx1, unsigned long idx2)
 868 asmlinkage long sys_finit_module(int fd, const char __user *uargs, int flags);
+869 asmlinkage int sys_prt_to_rbuf(char *s);
\end{verbatim}
編集により,\verb|sys_prt_to_rbuf|のプロトタイプ宣言がなされる.
\subsection{システムコール番号の定義}
システムコール番号を定義する.
%システムコール関数と,システムコール番号との対応づけが書かれているファイルを編集する.
本手順書では,\verb|/home/username/git/linux-stable/arch/|\\
\verb|x86/syscalls/syscall_64.tbl|を以下のように編集する.

\begin{verbatim}
 324 315    common    sched_getattr     sys_sched_getattr
 325 316    common    renameat2	        sys_renameat2
+326 317    common    sys_prt_to_rbuf   sys_prt_to_rbuf
\end{verbatim}
編集により,$317$番が\verb|sys_prt_to_rbuf|のシステムコール番号として定義される.
\subsection{Makefileの編集}
Makefileを編集する.本手順書では,\verb|/home/username/git/linux-stable/kernel/Makefile|を以下のように編集する.
\begin{verbatim}
  5 obj-y     = fork.o exec_domain.o panic.o \
  6             cpu.o exit.o itimer.o time.o softirq.o resource.o \
  7             sysctl.o sysctl_binary.o capability.o ptrace.o timer.o user.o \
  8             signal.o sys.o kmod.o workqueue.o pid.o task_work.o \
  9             extable.o params.o posix-timers.o \
 10             kthread.o sys_ni.o posix-cpu-timers.o \
 11             hrtimer.o nsproxy.o \
 12             notifier.o ksysfs.o cred.o reboot.o \
-13             async.o range.o groups.o smpboot.o
+13             async.o range.o groups.o smpboot.o prt_to_rbuf.o
\end{verbatim}
編集後,\verb|prt_to_rbuf.c|が,カーネルのコンパイルに含まれる.
\subsection{カーネルの再構築}

\subsubsection{概要}
本節では,カーネル再構築の手順について述べる.
\subsubsection{.configファイルの作成}
\verb|.config|ファイルを作成する.これは,カーネルの設定ファイルである.以下のコマンドを実行する.
\begin{description}
\item[] \verb|$ make defconfig|
\end{description}
実行後,\verb|x86_64_defconfign|ファイルに基づき,\verb|.config|ファイルが\verb|/home/username/git/linux-stable|以下に生成される.
\subsubsection{カーネルのコンパイル}
システムコールを追加したカーネルをコンパイルする.以下のコマンドを実行する.
\begin{description}
\item[] \verb|$ make bzImage -j8|
\end{description}
実行後,\verb|/home/username/git/linux-stable/arch/x86/boot|以下に\verb|bzImage|という名前の圧縮カーネルイメージが,\verb|/home/username/git/linux-stable|以下に\verb|System.map|というファイルが生成される.上記コマンドに設定されている\verb|-j|オプションは,同時に実行できるジョブ数を指定している.
\subsubsection{カーネルのインストール}\label{kerin}
コンパイルしたカーネルのインストールを行う.以下のコマンドを実行する.
\begin{description}
\item[] \verb|$ sudo cp /home/username/git/linux-stable/arch/x86/boot/bzImage|
\item[] \verb|/boot/vmlinuz-3.15.0-linux|
\item[] \verb|$ sudo cp /home/username/git/linux-stable/System.map|
\item[] \verb|/boot/System.map-3.15.0-linux|
\end{description}
実行後,カーネルをコンパイルした際に生成された,\verb|bzImage|と\verb|System.map|が,\verb|vmlinuz-3.15.0-linux|と\verb|System.map-3.15.0-linux|という名前で\verb|/boot|以下にコピーされ,カーネルのインストールが完了する.
\subsubsection{カーネルモジュールのコンパイル}
カーネルモジュールをコンパイルする.以下のコマンドを実行する.
\begin{description}
\item[] \verb|$ make modules|
\end{description}
実行後,カーネルモジュールがコンパイルされる.
\subsubsection{カーネルモジュールのインストール}\label{kermin}
コンパイルしたカーネルモジュールのインストールする.以下のコマンドを実行する.
\begin{description}
\item[] \verb|$ make modules_install|
\end{description}
実行後,コンパイルしたカーネルモジュールがインストールされる.このとき,端末に実行結果が出力されるが,その最下行には以下のように出力される.
\begin{description}
\item[] \verb|DEPMOD 3.15.0|
\end{description}
この,\verb|DEPMOD|以後に続く\verb|3.15.0|という部分は,カーネルモジュールがインストールされたディレクトリ名であり,環境によって異なる.このデレクトリ名は次の操作で用いるため,控えておく.
\subsubsection{初期RAMディスクの作成}
初期RAMディスクを作成する.本手順書では,以下のコマンドを実行する.このとき,\ref{kermin}項で控えておいたディレクトリ名を引数として与える.
\begin{description}
\item[] \verb|$ sudo update-initramfs -c -k 3.15.0|
\end{description}
実行により,初期RAMディスクが作成される.
\subsubsection{ブートローダの設定}
ブートローダの設定と再起動を行う.本手順書における環境では,ブートローダはGRUB2が用いられている.GRUB2の設定ファイルは\verb|/boot/grub/grub.cfg|であるが,このファイルは直接編集してはいけない.\verb|/etc/grub.d|以下にあるスクリプトを基に,この設定ファイルが生成されるので,このディレクトリにスクリプトを追加した後,コマンドを用いて設定ファイルを書き換える.その後,計算機を再起動する.手順を以下に述べる.
\begin{enumerate}
  \item 追加するスクリプトの作成\\
   追加するスクリプトを作成する.スクリプトのファイル名は,本手順書では\verb|/etc/grub.d|にある\verb|README|ファイルの説明に沿うように\verb|11_linux-3.15.0|とする.このスクリプトの内容を以下に示す.
    
\begin{verbatim}
1 #!/bin/sh -e
2 echo "Adding my custom Linux to GRUB2"
3 cat << EOF
4 menuentry "My custom Linux" {
5 set root=(hd0,1)
6 linux /vmlinuz-3.15.0-linux ro root=/dev/sda2 quiet
7 initrd /initrd.img-3.15.0
8 }
9 EOF
\end{verbatim}
\begin{description}
\item[4行目] mementry "My custom Linux"\\
ブートローダのカーネル選択画面に表示される名前を設定している.ここではMy custom Linuxとしている.
\item[5行目] set root=(hd0,1)\\
rootの値を設定している.ここで設定した場所に/bootが割り当てられる.hd0はHDDの番号,その右の1はパーティション番号である.
\item[6行目] linux /vmlinuz-3.15.0-linux ro root=/dev/sda2 quiet\\
  \begin{description}
  \item[] linux /vmlinuz-3.15.0-linux \\
    指定したファイルから,Linuxカーネルイメージを読み込んでいる.ここでは\ref{kerin}でインストールしたファイルを指定している.
  \item[] ro root=/dev/sda2 \\
    ルートファイルシステムと,そのマウント方法を指定している.ここでは\verb|/dev/sda2|に\verb|/root|を割り当て,読み込み専用でマウントするデバイスとして指定している.
  \item[] quiet \\
    カーネル起動時におけるメッセージ出力の省略を指定している.
  \end{description}
\item[7行目] initrd /initrd.img-3.15.0 \\
  読み込む初期RAMディスクを指定している.
\end{description}

\item 実行権限の付与\\
  作成したスクリプトに実行権限を付与する.以下のコマンドを実行する.\\
  \verb|$ sudo chmod +x /etc/grub.d/11_linux-3.15.0|\\
  実行により,11\_linux-3.15.0に実行権限が付与される.\\
\item ブートローダ設定ファイルの再生成\\
  ブートローダの設定ファイルを再生成する.以下のコマンドを実行する.\\
  \verb|$ sudo update-grub|\\
  実行により,ブートローダの設定ファイルが再生成される.そして,システムコールを追加したカーネルで起動できるようになる.\\
\end{enumerate}
\subsubsection{再起動}
再起動を行う.以下のコマンドを実行する.
\begin{verbatim}
$ sudo reboot
\end{verbatim}
  実行により,計算機が再起動する.このとき,カーネル選択画面でMy custom Linuxがエントリに追加されているので,それを選択し,起動する.
\section{動作テスト}
\subsection{概要}
本章では,追加したシステムコールの動作テストについて述べる.プログラムを用いて,追加したシステムコールを呼び出し,その振る舞いが意図したものであるか確認する.以降では,動作テスト用プログラムの作成と,動作テストの手順について,順に節を設け述べる.
\subsection{動作テスト用プログラムの作成}
\verb|prt_to_rbuf|の動作テストに用いるプログラムを作成する.本手順書における動作テスト用プログラムを以下に示す.
\begin{verbatim}
 1 #include<stdio.h>
 2 #include<unistd.h>
 3
 4 int main(){
 5  char buf[128];
 6  long sys_num = 317;
 7  int ret;
 8  scanf("%[^\r\n]",buf);
 9  ret = syscall(sys_num,buf);
10  printf("ret:%d\n",ret);
11  return 0;
12}
\end{verbatim}
上記の動作テスト用プログラムは,端末から任意の文字列の入力を受けつけ,\verb|syscall|に第2引数として渡す.同時に,\verb|syscall|には317という値が第1引数として渡されているが,これは\verb|sys_prt_to_rbuf|のシステムコール番号である.\verb|syscall|によって\verb|sys_prt_to_rbuf|を呼び出し,戻り値を端末に出力して終了する.
\subsection{動作テストの手順}
追加したシステムコールの動作テストの手順を以下に示す.
\begin{enumerate}
\item 動作テスト用プログラムの実行
\item dmesgコマンドの実行
\end{enumerate}
まず動作テストプログラムを実行する.本手順書における動作テスト用プログラムは,実行時に任意の文字列の入力を要求する.今回は,"under the water"と入力する.このとき,追加したシステムコールが適切に動作していれば,カーネルのメッセージバッファに先ほどのの文字列"under the water"が書き込まれているはずである.そこで,\verb|dmesg|コマンドを用いる.\verb|dmesg|コマンドは,カーネルのメッセージバッファの内容を端末に出力する機能を持つ.このときの出力結果を以下に示す.
\begin{verbatim}
[12300.934433] under the water
[12300.934434] 18 charactor(s) outputed
\end{verbatim}
上記の結果により,指定した文字列"under the water"がカーネルバッファに書き込まれていることがわかる.なお,各ログの左に表示されている数字は,計算機が起動してから各ログがメッセージバッファに書き込まれた時点までに経過した秒数である.追加したシステムコールは,意図した動作をしている.
\section{おわりに}
本手順書では,Linuxカーネルに対して新たにシステムコールを追加する手順について述べた.さらに,追加したシステムコールが,意図した動作をしているかどうかを調べるテストの手順について述べた.

\bibliographystyle{ipsjunsrt}
\bibliography{mybibdata}
%\begin{thebibliography}{1}
%\bibitem{jobs} 
%\end{thebibliography}
\end{document}
