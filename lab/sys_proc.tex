\documentclass[12pt]{jsarticle}
\usepackage[dvipdfmx]{graphicx}
\textheight = 25truecm
\textwidth = 18truecm
\topmargin = -1.5truecm
\oddsidemargin = -1truecm
\evensidemargin = -1truecm
\marginparwidth = -1truecm

\def\theenumii{\Alph{enumii}}
\def\theenumiii{\alph{enumiii}}
\def\labelenumi{(\theenumi)}
\def\labelenumiii{(\theenumiii)}
\def\theenumiv{\roman{enumiv}}
\def\labelenumiv{(\theenumiv)}
\usepackage{comment}

%%%%%%%%%%%%%%%%%%%%%%%%%%%%%%%%%%%%%%%%%%%%%%%%%%%%%%%%%%%%%%%%
%% sty/ にある研究室独自のスタイルファイル
\usepackage{jtygm}  % フォントに関する余計な警告を消す
\usepackage{nutils} % insertfigure, figref, tabref マクロ

\def\figdir{./figs} % 図のディレクトリ
\def\figext{pdf}    % 図のファイルの拡張子

\begin{document}
%%%%%%%%%%%%%%%%%%%%%%%%%%%%
%% 表題
%%%%%%%%%%%%%%%%%%%%%%%%%%%%
\begin{center}
{\LARGE 資料タイトル}
\end{center}

\begin{flushright}
  2018/4/10\\
  吉田 修太郎
\end{flushright}
%%%%%%%%%%%%%%%%%%%%%%%%%%%%
%% 概要
%%%%%%%%%%%%%%%%%%%%%%%%%%%%
\section{はじめに}
本手順書では,Linuxカーネルに対して新たにシステムコールを追加するための手順について述べる.本手順書において実装するのは,任意の文字列をカーネルバッファに書き込む機能を持つシステムコールである.以降では,実装環境,実装したシステムコールの概要,実装の手順,動作テストについて,順に章立ててそれぞれの詳細を述べる.

\section{実装環境}
本手順書における実装環境を下表に示す.
\begin{table}[h!]
  \begin{center}
    \caption{実装環境}%\label{tab:time_range_ratio}
    %\ecaption{Frequency of task ocurrences.}
    \begin{tabular}{r|r}
      \hline\hline
      OS & Debian 7.11 \\
      \hline
      カーネル & Linux カーネル 3.15.0 \\
      \hline
      CPU & Intel(R) Core(TM) i7-4770 \\
      \hline
      メモリ & 16GB\\
      \hline
    \end{tabular}
  \end{center}
\end{table}

\section{実装するシステムコールの概要}
ここでは,本手順書において実装するシステムコールの概要について述べる.
\begin{description}
\item[形式] asmlinkage int sys\_prt\_to\_rbuf(char *s)
\item[引数] char *s: 出力する文字列のポインタ
\item[戻り値] カーネルバッファに書き込んだ文字数
  %\begin{description}
  %\item[成功]:カーネルバッファに書き込んだ文字数
  %\item[失敗]:-1
  %\end{description}
\item[機能] 引数として受け取った文字列をカーネルバッファに書き込む
  
\end{description}
\section{実装の手順}
\subsection{概要}
本章では,システムコール実装の手順について述べる.以降では,ソースコードの作成,プロトタイプ宣言,システムコール番号の定義,Makefileの編集およびカーネルの再構築についてそれぞれについて節を設けて詳細に述べる.
\subsection{ソースコードの作成}
ここでは,実装するシステムコールのソースコードの作成について述べる.具体的な記述例として,本手順書3章において示したシステムコールの,ソースコードを以下に示す.なお,このソースコードはC言語で記述されている.
\begin{verbatim}
#include <linux/kernel.h>
#include <linux/syscalls.h>

asmlinkage int sys_prt_to_rbuf(char *s){
  int ret;
  ret = printk(KERN_INFO "%s\n",s);
  printk("%d charactor(s) outputed\n",ret);
  return ret;
}
\end{verbatim}
\subsection{プロトタイプ宣言}
ここでは,プロトタイプ宣言の手順について述べる.システムコール関数のプロトタイプ宣言が,まとめて書かれているヘッダファイルを探し,編集する.本手順書では,以下のファイルを編集する.
\begin{itemize}
\item \slash{}home\slash{}git\slash{}linux-stable\slash{}include\slash{}linux\slash{}syscalls.h
\end{itemize}
本手順書では,このヘッダファイルの末尾に以下の行を追加する.
\begin{itemize}
\item \slash{}home\slash{}git\slash{}asmlinkage int prt\_to\_rbuf(char *s);
\end{itemize}
\subsection{システムコール番号の定義}
ここでは,システムコール番号を定義する手順について述べる.システムコール関数と,システムコール番号との対応づけが書かれているファイルを探し,編集する.本手順書では,以下のファイルを編集する.
\begin{itemize}
\item \slash{}home\slash{}git\slash{}linux-stable\slash{}arch\slash{}x86\slash{}syscalls\slash{}syscall\_64.tbl
\end{itemize}
このファイルの内容の一部を抜粋し,以下に示す.
\begin{verbatim}
#
# 64-bit system call numbers and entry vectors
#
# The format is:
# <number> <abi> <name> <entry point>
#
# The abi is "common", "64" or "x32" for this file.
#
0	common	read			sys_read
1	common	write			sys_write
2	common	open			sys_open
3	common	close			sys_close
               ~ (中略) ~
314	common	sched_setattr		sys_sched_setattr
315	common	sched_getattr		sys_sched_getattr
316	common	renameat2		sys_renameat2

#
# x32-specific system call numbers start at 512 to avoid cache impact
# for native 64-bit operation.
#
512	x32	rt_sigaction		compat_sys_rt_sigaction
513	x32	rt_sigreturn		stub_x32_rt_sigreturn
514	x32	ioctl			compat_sys_ioctl
               ~ (中略) ~
540	x32	process_vm_writev	compat_sys_process_vm_writev
541	x32	setsockopt		compat_sys_setsockopt
542	x32	getsockopt		compat_sys_getsockopt
(EOF)
\end{verbatim}
上記のファイル内容の先頭付近に,このファイルのフォーマットは<number><abi><name><entry point>であるという旨が記述されている.それぞれの要素についての簡単な説明を以下に示す.
\begin{description}
\item[number] システムコール番号
\item[abi] Applicatioin Binary Interface
\item[name] 関数名
\item[entry point] 関数が 
\end{description}
このフォーマットに従い,実装したいシステムコールをこのファイルに追記する.ただし,システムコール番号は,システムコール呼出しの際に関数の特定に使用されるため,他の関数と重複してはならない.本資料では,以下のように設定する.
\begin{description}
\item[number] 317
\item[abi] common
\item[name] sys\_prt\_to\_rbuf
\item[entry point] sys\_prt\_to\_rbuf 
\end{description}
上記の場合におけるファイルへの記入例(変更した部位とその前後数行のみ抜粋)を以下に示す.先頭に+を置いて示した行が,追加された行である.
\begin{verbatim}
 314	common	sched_setattr		sys_sched_setattr
 315	common	sched_getattr		sys_sched_getattr
 316	common	renameat2		sys_renameat2
+317     commoc  sys_prt_to_rbuf         sys_prt_to_rbuf

#
# x32-specific system call numbers start at 512 to avoid cache impact
# for native 64-bit operation.
#
\end{verbatim}
\subsection{Makefile編集}
ここでは,Makefileの編集について述べる.今回編集するMakefileを以下に示す.
\begin{itemize}
\item \slash{}home\slash{}git\slash{}linux-stable\slash{}kernel\slash{}Makefile
\end{itemize}
makeコマンドは,このファイルの内容に基づいて実行されるため,今回追加したシステムコール関数をコンパイルするためには,ここにその処理を追記する必要がある.具体的には,Makefileの先頭付近に記述されている,各システムコール関数のオブジェクトファイルが代入されるobj-yという変数に対して,新たに作成したシステムコールのオブジェクトファイルも代入されるように追記する.本手順書における,このMakefileの編集内容(編集した部分のみ抜粋)を以下に示す.なお,ここでは削除した行の先頭に-を,追加した行の先頭に+を挿入している.

\begin{verbatim}
 #
 # Makefile for the linux kernel
 #
 
 obj-y     = fork.o exec_domain.o panic.o \
             cpu.o exit.o itimer.o time.o softirq.o resource.o \
             sysctl.o sysctl_binary.o capability.o ptrace.o timer.o user.o \
             signal.o sys.o kmod.o workqueue.o pid.o task_work.o \
             extable.o params.o posix-timers.o \
             kthread.o sys_ni.o posix-cpu-timers.o \
             hrtimer.o nsproxy.o \
             notifier.o ksysfs.o cred.o reboot.o \
-            async.o range.o groups.o smpboot.o
+            async.o range.o groups.o smpboot.o prt_to_rbuf.o
\end{verbatim}

\subsection{カーネルの再構築}
\subsubsection{.configファイルの作成}
カーネルを再構築するにあたり,はじめに,.configファイルを作成する.これは,カーネルの設定ファイルである.
\subsubsection{カーネルのコンパイル}
\subsubsection{カーネルのインストール}
\subsubsection{カーネルモジュールのコンパイル}
\subsubsection{カーネルモジュールのインストール}
\section{動作テスト}
\subsection{概要}
本章では,実装したシステムコールの動作テストについて述べる.以降では,動作テスト用プログラムの準備と,動作テストの手順について,順に節を設け述べる.
\subsection{動作テスト用プログラムの準備}
prt\_to\_rbufの動作テストに用いるプログラムを作成する.本手順書における動作テスト用プログラムを以下に示す.
\begin{verbatim}
#include<stdio.h>
#include<unistd.h>

int main(){
  char buf[128];
  long sys_num = 317;
  int ret;
  scanf("%[^\r\n]",buf);
  ret = syscall(sys_num,buf);
  printf("ret:%d\n",ret);
  return 0;
}
\end{verbatim}
\subsection{動作テストの手順}
動作テストの手順を以下に示す.
\begin{enumerate}
\item 動作テスト用プログラムの実行
\item dmesgコマンドの実行
\end{enumerate}

\section{おわりに}

本資料では

\bibliographystyle{ipsjunsrt}
\bibliography{mybibdata}

\end{document}
