\documentclass[fleqn, 14pt]{extarticle}
\usepackage{sty/reportForm}
\usepackage[utf8]{inputenc}
\usepackage[T1]{fontenc}
\usepackage{fixltx2e}
\usepackage{graphicx}
\usepackage{longtable}
\usepackage{float}
%\usepackage{wrapfig}
\usepackage[normalem]{ulem}
\usepackage{textcomp}
\usepackage{marvosym}
\usepackage{wasysym}
\usepackage{latexsym}
\usepackage{amssymb}
\usepackage{amstext}
\usepackage{hyperref}
\tolerance=1000
\subtitle{(2018年04月2日$\sim$2018年4月20日)}
\usepackage{sty/strike}
\setcounter{section}{-1}
\author{乃村研究室B4\\吉田 修太郎}
\date{2018年4月23日}
\title{記録書 No.2}
\hypersetup{
  pdfkeywords={},
  pdfsubject={},
  pdfcreator={Emacs 24.3.1 (Org mode 8.0.3)}}
\begin{document}

\maketitle
\section{前回ミーティングからの指導・指摘・学んだ事項}
\label{sec1}
\begin{enumerate}
\item 誰かの話を聞くときは,後で質問やコメントができるようにする.\\
  \hfill[4/4, 104号室,乃村先生]
\end{enumerate}

\section{実績}
\label{sec2}
\subsection{研究関連}
\label{sec2-1}
\begin{enumerate}
\item 2018年度B4新人研修課題に関する項目
  \hfill
  \label{sec2-1-1}
  \begin{enumerate}
  \item Debianのインストール
    \hfill
    \label{sec2-1-1-enum1}
    (100%,+100%)
  \item Linuxカーネルの再構築
    \hfill
    \label{sec2-1-1-enum2}
    (100%,+100%)
  \item システムコール実装の手順書作成
    \hfill
    \label{sec2-1-1-enum3}
    (80%,+80%)
  \item RubyによるSlackBotプログラムの作成
    \hfill
    \label{sec2-1-1-enum4}
    (15%,+15%)
  \item RubyによるSlackBotプログラムの報告書,仕様書作成
    \hfill
    \label{sec2-1-1-enum5}
    (0%,+0%)
  \end{enumerate}
\end{enumerate}

\subsection{研究室関連}
\label{sec2-2}
\begin{enumerate}
\item 平成30年度新B4ガイダンス
  \hfill
  \label{sec2-2-enum1}
  (4/2)
\item 平成30年度新B4歓迎会
  \hfill
  \label{sec2-2-enum1}
  (4/2)
\item 第349回New打ち合わせ
  \hfill
  \label{sec2-2-enum2}
  (4/3)
\item 平成30年度新B4向け Web勉強会
  \hfill
  \label{sec2-2-enum3}
  (4/3)
\item 平成30年度新B4向け rbenv勉強会
  \hfill
  \label{sec2-2-enum4}
  (4/3)
\item 平成30年度新B4向け Git勉強会
  \hfill
  \label{sec2-2-enum5}
  (4/4)
\item 乃村研お花見
  \hfill
  \label{sec2-2-enum6}
  (4/4)
\item 第350回New打ち合わせ
  \hfill
  \label{sec2-2-enum6}
  (4/18)
\item 乃村研ミーティング
  \hfill
  \label{sec2-2-enum6}
  (4/19)
\end{enumerate}

\subsection{大学関連}
\label{sec2-3}
\begin{enumerate}
\item 非手続き型言語
  \hfill
  \label{sec2-3-enum1}
  (4/9,10,16,17)
\end{enumerate}

\section{詳細および反省・感想}
\label{sec3}
\subsection{研究関連}
\label{sec3-1}
\begin{itemize}
\item[(\ref{sec2-1-1-enum4})] 現在,RubyによるSlackBotプログラムの作成に取り組んでいる.SlackBotとは,Slackというチャットツールで利用できるBotのことである.また,今回はフレームワークとしてSinatraを用いる.Sinatraとは,Rubyで作成されたオープンソースのwebアプリケーションフレームワークである.Rubyを用いてのコーディングは今回が初めてであり,Sinatraを利用することもまた初めてであるため,それぞれの仕様をよく調べながらコーディングを進める.
\end{itemize}

\subsection{研究室関連}
\label{sec3-2}
\begin{itemize}
\item[(\ref{sec2-2-enum5})] 本勉強会では,Gitの概要,その仕組み,および使い方について学んだ.これまでほとんどGitを利用することがなかったが,これを機に積極的にGitを利用し,Gitを十分に使いこなせるようになる.まずは,これまでクラウド上に最新のファイルのみをアップロードして管理していた文書について,今後はその管理をGitによって行う.
\end{itemize}

\section{今後の予定}
\label{sec4}
\subsection{研究関連}
\label{sec4-1}
\begin{enumerate}
\item RubyによるSlackBotプログラムの作成
  \hfill
  \label{sec4-1-enum1}
  (4/27)
\item RubyによるSlackBotプログラムの報告書,仕様書作成
  \hfill
  \label{sec4-1-enum2}
  (4/27)
\end{enumerate}

\subsection{研究室関連}
\label{sec4-2}
\begin{enumerate}
\item 全体ミーティング
  \hfill
  \label{sec4-2-enum1}
  (4/23)
\item 第351回New打ち合わせ
  \hfill
  \label{sec4-2-enum1}
  (4/27)
\item 乃村研ミーティング
  \hfill
  \label{sec4-2-enum1}
  (5/14)
%\item 第36回乃村杯
%  \hfill
%  \label{sec4-2-enum2}
%  (5/14)
\end{enumerate}

\subsection{大学関連}
\label{sec4-3}
\begin{enumerate}
\item 非手続き型言語 
  \hfill
  \label{sec4-3-enum1}
  (4/23, 24,5/1,7,14,15)
\end{enumerate}

\subsection{その他}
\label{sec4-3}
\begin{enumerate}
\item  Patchwork Okayama -2018
  \hfill
  \label{sec4-4-enum1}
  (5/11)
\end{enumerate}

%\section{その他}

\end{document}
