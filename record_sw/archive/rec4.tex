\documentclass[fleqn, 14pt]{extarticle}
\usepackage{sty/reportForm}
\usepackage[utf8]{inputenc}
\usepackage[T1]{fontenc}
\usepackage{fixltx2e}
\usepackage{graphicx}
\usepackage{longtable}
\usepackage{float}
%\usepackage{wrapfig}
\usepackage[normalem]{ulem}
\usepackage{textcomp}
\usepackage{marvosym}
\usepackage{wasysym}
\usepackage{latexsym}
\usepackage{amssymb}
\usepackage{amstext}
\usepackage{hyperref}
\tolerance=1000
\subtitle{(2018年04月23日$\sim$2018年5月23日)}
\usepackage{sty/strike}
%%%%%%%%%%%%%%%%%%%%%%%%%%%%
%\usepackage{jtygm}
%\usepackage{nutils}
%%%%%%%%%%%%%%%%%%%%%%%%%%%%
\setcounter{section}{-1}
\author{乃村研究室B4\\吉田 修太郎}
\date{2018年5月24日}
\title{記録書 No.4}
\hypersetup{
  pdfkeywords={},
  pdfsubject={},
  pdfcreator={Emacs 24.3.1 (Org mode 8.0.3)}}
\begin{document}
\maketitle
\section{前回ミーティングからの指導・指摘・学んだ事項}
\label{sec1}
\begin{enumerate}
\item 資料では,指示代名詞を避ける.\\
  \hfill[4/23, 第10講義室,谷口先生]
\item 資料では,初出の用語について説明する.\\
  \hfill[4/23, 第10講義室,谷口先生]
\item 資料では,図や表を用いて説明する.\\
  \hfill[4/27, 102号室,谷口先生,乃村先生]
\end{enumerate}

\section{実績}
\label{sec2}
\subsection{研究関連}
\label{sec2-1}
\begin{enumerate}
\item 2018年度B4新人研修課題に関する項目
  \hfill
  \label{sec2-1-1}
  \begin{enumerate}
  \item システムコール実装の手順書作成
    \hfill
    \label{sec2-1-1-enum1}
    (100%,+20%)
  \item RubyによるSlackBotプログラムの作成
    \hfill
    \label{sec2-1-1-enum2}
    (100%,+85%)
  \item RubyによるSlackBotプログラムの報告書,仕様書作成 \\
    \hfill
    \label{sec2-1-1-enum3}
    (100%,+100%)
  \end{enumerate}
\item 研究テーマに関する項目
  \label{sec2-1-2}
  \begin{enumerate}
  \item Mintのインストール
    \label{sec2-1-2-enum1}
    \hfill
    (90\% +90\%)
  \end{enumerate}
\end{enumerate}

\subsection{研究室関連}
\label{sec2-2}
\begin{enumerate}
\item SWLAB全体ミーティング
  \hfill
  \label{sec2-2-enum1}
  (4/23)
\item 株式会社クレオフーガ 訪問
  \hfill
  \label{sec2-2-enum2}
  (4/26)
\item 第351回New打ち合わせ
  \hfill
  \label{sec2-2-enum3}
  (4/27)
\item 乃村研ミーティング
  \hfill
  \label{sec2-2-enum4}
  (5/14)
\item 第36回乃村杯
  \hfill
  \label{sec2-2-enum5}
  (5/14)
\item 第352回New打ち合わせ
  \hfill
  \label{sec2-2-enum6}
  (5/17)
\item 岡山駅南地下道ビジョン 見学
  \hfill
  \label{sec2-2-enum7}
  (5/22)
\end{enumerate}

\subsection{大学関連}
\label{sec2-3}
\begin{enumerate}
\item 非手続き型言語
  \hfill
  \label{sec2-3-enum1}
  (4/24,5/1,7,14,21,22)
\end{enumerate}

\subsection{その他}
\label{sec2-4}
\begin{enumerate}
\item GitHub Patchwork Okayama -2018
  \hfill
  \label{sec2-4-enum1}
  (5/11)
\end{enumerate}

\section{詳細および反省・感想}
\label{sec3}
%\subsection{研究関連}
%\label{sec3-1}
%\begin{itemize}
%\item[(\ref{sec2-2-enum1})] 株式会社クレオフーガは,オーディオストックというサービスを運営している会社である.オーディオストックでは,音楽や音声などのデータをライセンスと一緒に販売している.会員登録すれば誰でもこれらを購入することができ,その音楽や音声を公に使用できる.さらに,会員は音楽や音声の販売も可能である.今回は主にこのオーディオストックについてのお話を伺い,また,同サービスの運営コストの削減についての議論があった.
%\end{itemize}

\subsection{研究室関連}
\label{sec3-2}
\begin{itemize}
\item[(\ref{sec2-2-enum1})] 株式会社クレオフーガは,オーディオストックというサービスを運営している会社である.オーディオストックでは,音楽や音声などのデータをライセンスと一緒に販売している.会員登録すれば誰でもこれらを購入することができ,その音楽や音声を公に使用できる.さらに,会員は音楽や音声の販売も可能である.今回は主にこのオーディオストックについてのお話を伺い,また,同サービスの運営コストの削減についての議論があった.
\item[(\ref{sec2-2-enum5})] 岡山駅南地下道ビジョンとは,株式会社ビザビが管理,運営する,岡山一番街とイオンモール岡山を繋ぐ地下道に設置されたデジタルサイネージ群である.デジタルサイネージとは,ディスプレイやプロジェクタ等によるデジタルな情報板,広告板である.今回の見学では,株式会社ビザビの方から岡山駅南地下道ビジョンに関するお話を伺い,デジタルサイネージの内側を見せて頂いた.また,今後デジタルサイネージを利用する機会を頂ける可能性があるため,広告に代わる新たなコンテンツについて考え,議論を行う予定である.
\end{itemize}

\section{今後の予定}
\label{sec4}
\subsection{研究関連}
\label{sec4-1}
\begin{enumerate}
\item Mintのインストール
  \hfill
  \label{sec4-1-enum1}
  (5/25)
\item Mintのインストール手順書作成
  \hfill
  \label{sec4-1-enum2}
  (6/7)
\end{enumerate}

\subsection{研究室関連}
\label{sec4-2}
\begin{enumerate}
\item SWLAB全体ミーティング
  \hfill
  \label{sec4-2-enum0}
  (5/24)
\item 第353回New打ち合わせ
  \hfill
  \label{sec4-2-enum1}
  (5/30)
  \item New開発打ち合わせ
  \hfill
  \label{sec4-2-enum2}
  (6/7)
\item 乃村研ミーティング
  \hfill
  \label{sec4-2-enum3}
  (6/13)

\end{enumerate}

\subsection{大学関連}
\label{sec4-3}
\begin{enumerate}
\item 非手続き型言語
  \hfill
  \label{sec4-3-enum1}
  (5/28,29)
\end{enumerate}

%\subsection{その他}
%\label{sec4-3}
%\begin{enumerate}
%\item  Patchwork Okayama -2018
%  \hfill
%  \label{sec4-4-enum1}
%  (5/28,29)
%\end{enumerate}

%\section{その他}
\bibliographystyle{ipsjunsrt}
\bibliography{mybibdata}

\end{document}
